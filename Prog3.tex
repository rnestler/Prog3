%%%%%%%%%%%%%%%%%%%%%%%%%
% Dokumentinformationen %
%%%%%%%%%%%%%%%%%%%%%%%%%
\documentclass[ngerman,english]{scrartcl}

\usepackage[iso,nocleanlook]{isodate}
\include{revision}
\title{Prog3 C++11 - Zusammenfassung}
\author{Raphael Nestler}
\date{\revisiondate ~ \revision ~ powered by \LaTeX}

%%%%%%%%%%%%%%%%%%%%%%%%%%%%%%%%%%%%%%%%%%%%%
% Standard projektübergreifender Header für
% - Makros 
% - Farben
% - Mathematische Operatoren 
%
% DORT NUR ERGÄNZEN, NICHTS LÖSCHEN
%%%%%%%%%%%%%%%%%%%%%%%%%%%%%%%%%%%%%%%%%%%%%  
\include{header/zusammenfassung}
\include{header/hyperref}

\usepackage{listings}

\newcommand{\includecode}{\begingroup
  \catcode`_=12 \docodelst}
\newcommand{\docodelst}[1]{
  \lstinputlisting[caption=\texttt{#1}]{#1}
  \endgroup
}

\lstset{
	tabsize=2,
	language=C++,
	breaklines=true,
	postbreak=\raisebox{0ex}[0ex][0ex]{\ensuremath{\hookrightarrow\space}},
	basicstyle=\footnotesize\ttfamily,
	keywordstyle=\bfseries\color{green!40!black},
	commentstyle=\itshape\color{purple!40!black},
	identifierstyle=\color{blue},
	stringstyle=\color{orange},
	morekeywords={for_each, decltype, constexpr}
}

\newcommand{\chapref}[1]{(siehe \ref{#1}~\nameref{#1})}

% Möglichst keine Ergänzungen hier, sondern in header.tex
\begin{document} 
\twocolumn
\section{Grundlagen}

\subsection{Variablen}
\begin{lstlisting}
int a{42};
int zero{};
\end{lstlisting}
When initializing with ''=`` we can use ''auto`` for the type
\begin{lstlisting}
auto i=5;
\end{lstlisting}

For long types we use ''alias``
\begin{lstlisting}
using input=std::istream_iterator<std::string>;
input in{std::cin}
input eof{};
\end{lstlisting}

\subsection{Built in Types}
\begin{itemize}
\item bool
\item char, unsigned char, wchar\_t, char16\_t, char32\_t
\item short, int, long, long long
\item unsigned short, unsigned int, unsigned long, unsigned long long
\item float, double, long double
\end{itemize}
\section{Iteratoren}
\subsection{std::vector}
\begin{lstlisting}
std::vector<X> v(6);
\end{lstlisting}
Iterieren:
\begin{lstlisting}
	for(size_t i=0; i<v.size(); ++i)
		std::cout << v[i];
	
	for(auto const i:v)
		std::cout << i;
	
	for(auto &j:v)
		j*=2;

	for(auto it=v.cbegin(); it != v.cend(); ++it)
		std::cout << *it;
\end{lstlisting}
Wenn möglich nicht verwenden, immer algorithmen

\subsection{Forward Iterator}

\subsection{Bidriectional Iterator}

\subsection{Random access Iterator}
Bidirectional but Allows indexing with $[]$



\section{Algorithmen}
\begin{itemize}
\item distance
\item copy / copy\_if
\item count / count\_if
\item \label{find} find / find\_if
\end{itemize}

\section{Containers}
\subsection{Übersicht}
\subsubsection{Sequence}
\begin{itemize}
	\item Reihenfolge bleibt erhalten
	\item find \chapref{find} in linearer Zeit
	\item array, vector, deque, list, forward\_list
\end{itemize}
\subsubsection{Associative}
\begin{itemize}
	\item Reihenfolge in sortierter Folge
	\item find \chapref{find} in logarithmischer Zeit
	\item map, multimap, set, multiset
\end{itemize}
\subsubsection{Hashed}
\begin{itemize}
	\item ohne Reihenfolge
	\item find \chapref{find} in konstanter Zeit
	\item unordered\_map, unordered\_multimap, \newline
		unordered\_set, unordered\_multiset
\end{itemize}
\subsubsection{Adaptors}
\begin{itemize}
	\item Provide a different interface for sequential containers. 
	\item stack
	\begin{itemize}
	\item limits to push / pop and top
	\item wraps deque, vector or list
	\end{itemize}
	\item queue
	\begin{itemize}
	\item push / pop / front
	\end{itemize}
	\item priority\_queue
	\begin{itemize}
	\item push / pop / top
	\item top() element is always smallest
	\end{itemize}
\end{itemize}
\subsection{Common API}
\begin{itemize}
	\item erease(iter)
	\item insert(iter,value)
	\item size(), empty()
	\item default-constructor
	\item copy-constructor
	\item equal-compare wenn Elemente gleicher Typ
	\item clear() $\rightarrow$ empty()
\end{itemize}
\subsubsection{Constructors}
	\begin{tabularx}{\columnwidth}{lX}
		Code & Bedeutung \\
		\hline
		C\{\} & Leerer Container besser als C() \\
		C\{e1,e2,\ldots,en\} & Mit \emph{initilizer list} initialisieren \\
		C\{n, e\} & Mit \emph{n} Kopien von \emph{e} initialisieren.  Besser C(n,e) verwenden, da sonst eine Mehrdeutigkeit ensteht, wenn \emph{n} ein gültiges Element des Containers ist (z.B. bei $vector<int>$) \\
		C\{ita, itb\} & Mit Elementen von iterator \emph{ita} bis ohne \emph{itb} initialisieren \\
	\end{tabularx}
	Beispiele:
\begin{lstlisting}
	std::vector<int> v{1,2,3,5,7,11};
	std::list<int>   l(5,42);
	std::deque<int>  q{v.begin(),v.end()};
\end{lstlisting}

\subsection{vector}
\begin{center}
\includegraphics[width=\linewidth]{./bilder/iterators}
\end{center}
\begin{itemize}
\item Bool vector is bad because of premature optimization
\end{itemize}


\subsection{deque}
\begin{itemize}
\item push\_front / pop\_front
\end{itemize}

\subsection{array}
\begin{itemize}
\item similar to C-array (fixed size), but keeps size
\item initialization needs \{\{\}\}
\end{itemize}

\subsection{list}
\begin{itemize}
\item double linked list
\item fast insert in any position
\end{itemize}

\subsection{set}
\begin{itemize}
\item own member functions for find and count which are more efficient
\end{itemize}
\begin{lstlisting}
#include <set>
#include <iostream>
void filtervowels(std::istream &in, std::ostream &out){
	 std::set<char> const vowels{'a','e','o','u','i','y'};
	 char c{};
	 while (in>>c)
	 	 if (! vowels.count(c))
	 	 	 out << c;
}
int main(){
	 filtervowels(std::cin,std::cout);
}
\end{lstlisting}

\subsection{map}
\begin{lstlisting}
#include <map>
#include <iostream>
#include <iterator>
int main(){
	 std::map<char,size_t> vowels
	
{{'a',0},{'e',0},{'i',0},{'o',0},{'u',0},{'y',0}};
	 char c;
	 while (std::cin >> c)
	 	 if (vowels.count(c))
	 	 	 ++vowels[c];
	 for(auto const &p:vowels)
	 	 std::cout << p.first << " = "<< p.second << '\n';
}
\end{lstlisting}

\section{Functions}
\label{Function}

\subsection{Normale Syntax}
\subsubsection{pass by value}
\begin{lstlisting}
RetType f(Type par);
\end{lstlisting}
\subsubsection{pass by const-ref}
\begin{lstlisting}
RetType f(Type const &par);
\end{lstlisting}
\subsubsection{pass by ref}
\begin{lstlisting}
RetType f(Type &par);
\end{lstlisting}

\subsection{Lambdas}
\label{Lambda}
\begin{lstlisting}
auto f = [](Type par)->RetType{};
\end{lstlisting}

\subsection{Functors}

\subsubsection{Standard Functors}
\begin{lstlisting}
#include <functional>
\end{lstlisting}
\begin{itemize}
\item arithmetic / logical
	\begin{itemize}
	\item plus (+)
	\item minus (-)
	\item divides (/)
	\item multiplies (*)
	\item modulus (\%)
	\item logical\_and (\&\&)
	\item logical\_or ($||$)
	\end{itemize}
	
	\item unary
	\begin{itemize}
	\item negate (-)
	\item logical\_not (!)
	\end{itemize}
	
	\item binary comparision
	\begin{itemize}
	\item less (<)
	\item less\_equal (<=)
	\item equal\_to (==)
	\item greater\_equal (>=)
	\item greater (>)
	\item not\_equal\_to (!=)
	\end{itemize}
\end{itemize}

\subsubsection{Own functors}
\begin{lstlisting}
#include <set>
#include <functional>
#include <algorithm>
#include <cctype>
#include <iterator>
#include <iostream>
struct caseless{
	 using string=std::string;
	 bool operator()(string const &l, string const &r){
	 	 return std::lexicographical_compare(
	 	 	 	 l.begin(),l.end(),r.begin(),r.end(),
	 	 	 	 [](char l,char r){
	 	 	 return std::tolower(l) < std::tolower(r);
	 	 });
	 }
};
int main(){
	 using std::string;
	 using caseless_set=std::multiset<string,caseless>;
	 using in=std::istream_iterator<string>;
	 caseless_set wlist{in{std::cin},in{}};
	 std::ostream_iterator<string> out{std::cout,"\n"};
	 copy(wlist.begin(),wlist.end(),out);
}
\end{lstlisting}

\subsection{Functional Programming}


\section{Templates}
\subsection{Functions}

\begin{itemize}
\item Function templates define a family of functions
with different template arguments
\item When you specify template arguments in angle
brackets, the template is ''instantiated`` for the
given arguments
\item Only needed, when template parameter is not
used as a function parameter type, or when
ambiguities exist.
\item Template functions can be overloaded as
template functions or regular functions
\end{itemize}

\begin{lstlisting}
template <typename T>
auto square(T value)->decltype(value*value){
	return value*value;
}
\end{lstlisting}
\begin{lstlisting}
namespace MyMin{
	template <typename T>
	T const& min(T const& a, T const& b){
		 return (a < b)? a : b ;
	}
}
\end{lstlisting}

\begin{itemize}
\item Compiler can automatically deduce the type of the template
\begin{lstlisting}
int i = 88;
min(i,42);
double pi = 3.1415;
double e = 2.7182;
min(e,pi);
\end{lstlisting}

\item If ambiguous, you need to specify the type
\begin{lstlisting}
//min(2,pi); // compile error
min(static_cast<double>(2),pi);
min<double>(2,pi);

\end{lstlisting}
\end{itemize}


\subsection{Variadic Templates}
\begin{lstlisting}
template <typename...ARGS>
void variadic(ARGS...args){
	 println(std::cout,args...);
}

template<typename Head, typename... Tail>
void println(std::ostream &out, Head const& head, Tail const& ...tail) {
	 out << head;
	 if (sizeof...(tail)) {
	 	 out << ", ";
	 }
	 println(out,tail...); //recurse on tail
}
\end{lstlisting}

\subsection{Prohibiting Overloads}
\begin{lstlisting}
template <typename T>
T const& min(T const& a, T const& b){
	 return (a < b)? a : b ;
}
template <typename T>
T * min(T * a, T * b)=delete; // disable for pointers
char const* min(char const* a, char const* b); // but enable for char const pointers
\end{lstlisting}


\subsection{Classes}
\begin{itemize}
\item Must always specify template arguments
\end{itemize}

\includecode{snippets/dynArray.h}

\subsection{SFINAE}
Schlechter passender match wird gewählt wenn perfekter match ein compile-fehler geben würde.

\section{Vererbung und dynamischer Polymorphismus}
\begin{lstlisting}
	class Base {};
	class Derived : public Base {
		Derived():Base{}{}
	};
\end{lstlisting}
\subsection{Multiple Inheritance}
\begin{lstlisting}
	class Base {};
	struct MixIn{};
	class MultipleBase : public Base, private MixIn {
		MultipleBase():Base{},MixIn{}
		{}
	};
\end{lstlisting}


\subsection{Visibility}
% eher tabelle?
\begin{itemize}
	\item public:
		members of base are visible and usable from derived class
	\item protected:
		members of base are usable from derived class
	\item private:
		members of base can not be accessed
\end{itemize}

\subsection{Converting operators}
\begin{lstlisting}
explicit
operator Typ() const {
	return ...
}
\end{lstlisting}

\subsection{Constructors}

\subsubsection{explicit}
verhindert implizite, automatische casts, bei Konstruktoren mit einem Parameter und Konvertierungsoperatoren.

\input{sections/cute.tex}
\section{Heap Memory}
\subsection{unique\_ptr}
\begin{lstlisting}
std::unique_ptr<X> factory(int i){
	 return std::unique_ptr<X>{new X{i}};
}
\end{lstlisting}
\begin{itemize}
	\item for unshared heap memory
	\item returned from a factory function
	\item can wrap to-be-freed pointers from C functions
\end{itemize}

\subsection{shared\_ptr}
\begin{lstlisting}
std::shared_ptr<A> A_factory(){
	return std::make_shared<A>(5,"hi",'a');
}
std::shared_ptr<std::ostream> os_factory(bool str){
	if (str)
		return std::make_shared<std::ostringstream>();
	else
		return std::make_shared<std::ofstream>("hello.txt");
}
\end{lstlisting}
\begin{itemize}
\item last $shared\_ptr$ handle destroyed will delete
allocated object
\item if instances of a class hierarchy are always
represented by a $shared\_ptr<base>$ but created
through $make\_shared<concrete>()$ the
destructor no longer needs to be virtual
\item $shared\_ptr$ memorizes concrete destructor for
deletion on construction time
\item $shared\_ptr$ can lead to object cycles no longer
cleared, because of circular dependency
\item need $weak\_ptr$ to break such cycles
\end{itemize}

\subsection{weak\_ptr}
\begin{itemize}
	\item avoid circular object dependencies with shared\_ptr
	\item keep "living" objects in aseparate data structure as shared\_ptr and relationships / dpendencies with weak\_ptr
\end{itemize}

\subsection{PIMPL}
\begin{lstlisting}
#include <memory>
#include <string>
class Wizard {
	 std::unique_ptr<class WizardImpl> pImpl;
public:
	 Wizard(std::string name);
	 ~Wizard(); // must declare dtor
	 std::string doMagic(std::string wish);
};
\end{lstlisting}
\begin{itemize}
	\item exported header file with class consisting of a ''Pointer to IMPLementation``
	\item Shield client code from implementation changes
\end{itemize}

 
\end{document}
