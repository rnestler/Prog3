\section{Templates}
\subsection{Functions}

\begin{itemize}
\item Function templates define a family of functions
with different template arguments
\item When you specify template arguments in angle
brackets, the template is ''instantiated`` for the
given arguments
\item Only needed, when template parameter is not
used as a function parameter type, or when
ambiguities exist.
\item Template functions can be overloaded as
template functions or regular functions
\end{itemize}

\begin{lstlisting}
template <typename T>
auto square(T value)->decltype(value*value){
	return value*value;
}
\end{lstlisting}
\begin{lstlisting}
namespace MyMin{
	template <typename T>
	T const& min(T const& a, T const& b){
		 return (a < b)? a : b ;
	}
}
\end{lstlisting}

\begin{itemize}
\item Compiler can automatically deduce the type of the template
\begin{lstlisting}
int i = 88;
min(i,42);
double pi = 3.1415;
double e = 2.7182;
min(e,pi);
\end{lstlisting}

\item If ambiguous, you need to specify the type
\begin{lstlisting}
//min(2,pi); // compile error
min(static_cast<double>(2),pi);
min<double>(2,pi);

\end{lstlisting}
\end{itemize}


\subsection{Variadic Templates}
\begin{lstlisting}
template <typename...ARGS>
void variadic(ARGS...args){
	 println(std::cout,args...);
}

template<typename Head, typename... Tail>
void println(std::ostream &out, Head const& head, Tail const& ...tail) {
	 out << head;
	 if (sizeof...(tail)) {
	 	 out << ", ";
	 }
	 println(out,tail...); //recurse on tail
}
\end{lstlisting}

\subsection{Prohibiting Overloads}
\begin{lstlisting}
template <typename T>
T const& min(T const& a, T const& b){
	 return (a < b)? a : b ;
}
template <typename T>
T * min(T * a, T * b)=delete; // disable for pointers
char const* min(char const* a, char const* b); // but enable for char const pointers
\end{lstlisting}


\subsection{Classes}
\begin{itemize}
\item Must always specify template arguments
\end{itemize}

\includecode{snippets/dynArray.h}
